\documentclass[10pts]{article}
\usepackage[utf8]{inputenc}
\usepackage{amsmath}
\makeatletter
\usepackage{mathtools}
\usepackage[utf8]{inputenc}
\usepackage[top=2cm, bottom=2cm, left=2cm, right=2cm]{geometry}
% Permet de définir les marge
\usepackage{amsfonts}
% Permet d'utiliser les commandes tels que \mathbb{R}
\usepackage{tikz}
\usepackage{amssymb}
\usepackage{algorithm}
\usepackage{algpseudocode}  
\usepackage{indentfirst}
% Marge de début de paragraphe
\usepackage{parskip}
\usepackage{multicol}
% Permet d'avoir plusieurs colonnes 
\usepackage{mathtools}
% pour utiliser PairedDelimiter sans utiliser \left \right
\usepackage{hyperref}
\renewcommand\not[1]{#1\xnot}
  \renewcommand{\notin}{\not\in}

\renewcommand{\baselinestretch}{2} 
\usepackage{listings}
\usepackage{xcolor}

\usepackage{graphicx}
\graphicspath{ {./images/} }

\usepackage{hyperref}
\usepackage[colorlinks=true, linkcolor=blue, urlcolor=blue, citecolor=blue]{hyperref}


\usepackage[round]{natbib}
\usepackage[colorlinks=true, citecolor=blue, urlcolor=blue, linkcolor=blue]{hyperref}

\lstset{
    language=Python,
    basicstyle=\ttfamily\small,
    keywordstyle=\color{blue},
    stringstyle=\color{red},
    commentstyle=\color{green},
    showstringspaces=false,
}

\DeclarePairedDelimiterX\paren[1]{(}{)}{#1}
% utiliser * pour des grosse parenthèse plutôt que \left \right 
% ex \paren*{\frac{1}{2}}

\DeclarePairedDelimiterX\bracket[1]{[}{]}{#1}
% même chose que l'autre d'avant mais pour des crochets

\usepackage{bm}
\usepackage{array}
\usepackage[utf8]{inputenc}
\usepackage{fancyhdr}
\pagestyle{fancy}
\fancyhf{} % clear all header and footer fields
\renewcommand{\headrulewidth}{0.4pt} % add line under header
\renewcommand{\footrulewidth}{0pt}  % no line under footer

% Define the headers
\fancyhead[C]{Katherine DEMERS} % Centered header

% Define the footers (optional)
\fancyfoot[C]{\thepage} % Centered footer with page number

\title{Understanding Diversity in Multi-Objective GFlowNets: A Systematic Study of Mechanisms and Metrics}
\author{Katherine Demers}
\date{October 2025}

\begin{document}

\maketitle
\section{Baseline Establishment}
\subsection{Project Goal}
Systematically dissect and understand what drives diversity in multi-objective GFlowNets,
comparing the effects of hypernetwork capacity, preference sampling strategies, and loss functions, while proposing novel GFlowNet-specific diversity metrics.

Our goals is to reproduce MOGFN-PC on the hypergrid task, implements all metrics and establish baseline diversity levels.

\subsection{Relevance of this project}
While existing MOGFN papers measure diversity (via metrics like pairwise distance),
there's limited understanding of what actually creates diversity, how to measure it properly and how to deliverately control diversity. Out expected contribution with this paper is a comprehensive empirical study providing actionable insights for practitioners and a suite of new diversity metrics tailored to GFlowNets.

\section{Introduction}

\section{Background \& Related Work}
\subsection{GFloNets}

\subsection{Diversity in Multi-Objective Optimization}
Quality-diversity algorithms \citep{pugh2016quality, 7959075} balance 
solution quality with behavioral diversity. Our work extends these principles 
to GFlowNets, introducing metrics that leverage the unique properties of 
generative flow networks \citep{bengio2021flow}...

\subsection{Factors affecting Diversity}

\section{Diversity Metrics for GFlowNets}
\subsection{Limitations of Existing Metrics}
While traditional MOO metrics \citep{zitzler1999multiobjective, deb2002fast} 
measure solution quality, they do not capture GFlowNet-specific characteristics.

\subsection{Proposed Metrics} 
We propose 14 novel metrics organized into five categories:

\textbf{Trajectory-Based Metrics.} These metrics exploit GFlowNet's sequential 
generation process. Our Trajectory Diversity Score (TDS) adapts edit distance 
\citep{levenshtein1966binary} to measure path diversity...

\subsection{Metric Validation}

\section{Experimental Design}
\subsection{Research Questions}

\subsection{Factors \& Levels}

\subsection{Tasks \& Baselines}

\subsection{Implementation Details}

\section{Results}
\subsection{Baseline Performance}

\subsection{Ablation Studies}
\subsubsection{Hypernetwork Capacity Effects}

\subsubsection{Preference Sampling Effects}

\subsubsection{Loss Function Effects}

\subsection{Interaction Analysis}

\subsection{Metric Analysis}

\section{Analysis \& Discussion}
\subsection{What Drives Diversity?}

\subsection{Practical Guidelines}

\subsection{Metric Recommendations}

\subsection{Limitations \& Future Work}

\section{Conclusions}

\bibliographystyle{plainnat}
\bibliography{references}

\end{document}
